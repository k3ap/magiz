\sklop{Osnove analize}

\snov{Obseg realnih števil, polnost}

Racionalna števila niso poln prostor, saj v njih obstajajo Cauchyjeva zaporedja
brez limit.
Primer takšnega zaporedja je recimo
\begin{align*}
  a_1 &= 1 \\
  a_2 &= 1.4 \\
  a_3 &= 1.41 \\
  a_4 &= 1.414 \\
  a_5 &= 1.4142 \\
  & \ldots
\end{align*}
torej racionalni približki korena $2$.
Zaporedje je Cauchyjevo, torej ima limito $L$ (za katero velja $L^2 = 2$ po
konstrukciji), vendar $L \notin \Q$:
Recimo nasprotno, da je $L = m / n$ za $m,n \in \N$ z $\gcd(m,n) = 1$.
Potem velja $2 = L^2 = m^2 / n^2$, iz česar dobimo $m^2 = 2 n^2$.
Sklepamo, da mora veljati tudi $m^2 = 4 k^2$ za nek $k \in \N$, saj praštevilski
delitelj kvadrata deli tudi osnovno število.
Torej je $n^2 = 2 k^2$, in podobno sklepamo, da $2$ deli $n$.
To je protislovje s predpostavko $\gcd(m,n) = 1$.

Realna števila lahko definiramo kot napolnitev $\Q$, a jo raje podamo z
Dedekindovimi rezi, saj je tako lažje pokazati, da je to urejeno polje (tj.~da
je polje z definirano relacijo $\le$, ki zadošča $a \le b \implies a + c \le b +
c$ in $0 \le a \land 0 \le b \implies 0 \le a \cdot b$).
Poleg tega v $\R$ zahtevamo še dodaten aksiom:
Vsaka neprazna podmnožica $\R$ mora imeti supremum v $\R$.

\begin{definicija}
  Podmnožica $A \subseteq \Q$ je \pojem{Dedekindov rez}, če velja
  \begin{itemize}
  \item $A \ne \varnothing$ in $A \ne \Q$,
  \item če $p \in A$, $q \in \Q$ ter $q < p$, je tudi $q \in A$,
  \item za vsak $p \in A$ obstaja $q \in A$, da $q > p$.
  \end{itemize}
  Množico vseh rezov imenujemo \pojem{realna števila} in označimo z $\R$.
\end{definicija}

\noindent
Med rezi na naraven način definiramo znane operacije:
\begin{itemize}
\item $A + B = \{ a + b \such a \in A, b \in B \}$,
\item $A \le B \iff A \subseteq B$,
\item za pozitivna $A, B$: $A \cdot B = \{ p \in \Q \such \exists a \in A, b \in
  B \velja p \le a \cdot b, a > 0, b > 0\}$,
\item za $A > 0$, $B < 0$: $A \cdot B = -(A \cdot (-B))$,
\item podobno za $A < 0, B > 0$ ter $A, B < 0$.
\end{itemize}
Preverimo lahko (a ne bomo, ker je dolgočasno), da te operacije ustrezajo
aksiomom urejenega polja.

\begin{trditev}
  $\R$ ustreza Dedekindovemu aksiomu.
\end{trditev}

\begin{proof}
  Naj bo $\mathcal{A} \subseteq \R$ neprazna in navzgor omejena.
  Definiramo $C = \bigcup \mathcal{A}$.
  To je očitno tudi rez, pokažimo še, da je supremum množice $\mathcal{A}$.
  Ker je $A \subseteq C$ za vse $A \in \mathcal{A}$, je $C$ zgornja meja za
  $\mathcal{A}$.
  Recimo, da je $D < C$ neko realno število.
  Potem obstaja $s \in C \setminus D$, in $A \in \mathcal{A}$, da $s \in A$.
  Potem je $A > D$ in $D$ ni zgornja meja za $\mathcal{A}$.
\end{proof}

\vprasanje{Definiraj $\R$ z Dedekindovimi rezi in pokaži, da ustreza
  Dedekindovemu aksiomu.}

\begin{posledica}
  $\Z$ v $\R$ ni navzgor omejena.
  Za poljubna pozitivna $a, b \in \R$ obstaja $n \in \N$, da je $na > b$.
  Slednjemu pravimo \pojem{arhimedska lastnost}.
\end{posledica}

\begin{proof}
  Če je $\Z$ navzgor omejena, obstaja natančna zgornja meja $\alpha$.
  Potem $\alpha-1$ ni zgornja meja, in obstaja $m \in \Z$, da je $m > \alpha -
  1$, oziroma $m + 1 > \alpha$.
  \protislovje{}

  Za drugo trditev opazimo, da je $na >b$ natanko tedaj, ko je $n > b/a$.
  Tak $n$ obstaja, saj $\Z$ ni navzgor omejena (torej tudi $\N \subseteq \Z$
  ni).
\end{proof}

\begin{posledica}
  $\Q$ je gosta v $\R$.
\end{posledica}

\begin{proof}
  Topologija v $\R$ je porojena z metriko $d(x,y) = \abs{x-y}$, torej je dovolj
  pokazati, da za vsaki realni $a < b$ obstaja racionalen $q$, za katerega velja
  $a < q < b$.
  Izberimo nek $q' \in b \setminus a$.
  Po definiciji Dedekindovega reza obstaja $q \in b$, ki je strogo večji od
  $q'$.
  Potem je $a < q < b$.
\end{proof}

\begin{opomba}
  Če je $a \in \Q$, bi se lahko zgodilo $a = q'$, zato moramo vzeti nek večji
  element.
\end{opomba}

\vprasanje{Pokaži, da $\Z$ v $\R$ ni navzgor omejena. Kaj je arhimedska
  lastnost?}

\begin{izrek}
  $\R$ je poln prostor.
\end{izrek}

\begin{proof}
  Naj bo $(x_n)_n$ Cauchyjevo zaporedje v $\R$.
  Definiramo
  \[
	L = \{ q \in \Q \such \text{$q$ je vsebovan v neskončno mnogo $x_n$} \}.
  \]
  To je navzgor omejena množica, ker je zaporedje $(x_n)_n$ navzgor omejeno.
  Po Dedekindovem aksiomu obstaja supremum $l$.
  Naj bo $\varepsilon > 0$ in $q \in L$ tak, da je $l - q <
  \frac{\varepsilon}{2}$.
  Ker je $q \in L$, obstaja podzaporedje $(x_{n_k})_k$ zaporedja $(x_n)_n$, da je
  $q \in x_{n_k}$ za vse $k$, oziroma $q < x_{n_k}$ za vse $k$.

  Recimo, da obstaja neskončno členov $(x_{n_k})_k$, ki so od $q$ oddaljeni za
  vsaj $\frac{\varepsilon}{2}$.
  Naj bo $d = \frac{\varepsilon}{2} - (l - q) > 0$.
  Obstaja racionalno število $q' \in (l, l+d)$, ki je po konstrukciji vsebovano
  v vseh členih $(x_{n_k})_k$, ki so od $q$ oddaljeni za vsaj
  $\frac{\varepsilon}{2}$; teh pa je po predpostavki neskončno, torej je $q' \in
  L$ in $l < q'$ ne mora biti natančna zgornja meja.
  \protislovje{}

  Torej je takšnih členov le končno mnogo, in obstaja $K \in \N$, da je $x_{n_k}
  - q < \frac{\varepsilon}{2}$ za vse $k \ge K$.
  Tedaj je
  \[
	\abs{x_{n_k} - l} \le \abs{x_{n_k} - q} + \abs{q - l}
	< \frac{\varepsilon}{2} + \frac{\varepsilon}{2}
	= \varepsilon
  \]
  za vse $k \ge K$, torej ima $(x_n)_n$ konvergentno podzaporedje.
  Ker je Cauchyjevo, je tudi samo konvergentno.
\end{proof}

\vprasanje{Pokaži: $\R$ je poln prostor.}

\snov{Številska zaporedja}

Zaporedje realnih števil je preslikava $a: \N \to \R$.
Namesto $a(n)$ pogosto pišemo $a_n$, za celotno zaporedje pa $(a_n)_n$.
Pravimo, da zaporedje \pojem{konvergira} proti številu $L \in \R$, če za vsak
$\varepsilon > 0$ obstaja $n_0 \in \N$, da za vse $n \ge n_0$ velja $\abs{a_n -
  L} < \varepsilon$.
Številu $L$ pravimo \pojem{limita} zaporedja $(a_n)_n$ in pišemo
\[
  L = \lim_{n \to \infty} a_n.
\]

\begin{trditev}
  Konvergentno zaporedje ima natanko eno limito.
\end{trditev}

\begin{proof}
  Naj bosta $L_1 \ne L_2$ limiti zaporedja $(a_n)_n$.
  Brez škode za splošnost lahko predpostavimo $L_1 > L_2$.
  Označimo $d = L_1 - L_2$.
  Potem za $\varepsilon = \nicefrac{d}{3}$ obstaja $n_0$, da so vsi členi
  zaporedja od $n_0$-tega dalje v $\varepsilon$-okolici obeh $L_1, L_2$, kar je
  protislovno, saj imata ti okolici prazen presek.
\end{proof}

\begin{trditev}
  \label{trditev:konvergentno-implicira-omejeno}
  Vsako konvergentno zaporedje je omejeno.
\end{trditev}

\begin{proof}
  Le končno mnogo členov je zunaj $1$-okolice limite.
\end{proof}

\begin{primer}
  Obrat ne velja; $a_n = (-1)^n$ je omejeno, a ni konvergentno.
\end{primer}

\begin{definicija}
  Število $s \in \R$ je \pojem{stekališče} zaporedja $(a_n)_n$, če v vsaki
  $\varepsilon$-okolici $s$ leži neskončno mnogo členov zaporedja.
\end{definicija}

Če je zaporedje konvergentno, je limita njegovo edino stekališče.
Obratno ne velja; zaporedje
\[
  a_{2n} = 1 \qquad a_{2n+1} = n
\]
ima natanko eno stekališče ($1$), a ni konvergentno.

\begin{izrek}
  Vsako omejeno zaporedje ima stekališče.
\end{izrek}

\begin{proof}
  Naj bo
  \[
	U = \{ u \in \R \such \text{$a_n < u$ za kvečjemu končno mnogo $n \in \N$}
	\}.
  \]
  Ta množica je očitno omejena in neprazna, saj je $(a_n)_n$ omejeno.
  Torej obstaja $a = \sup U$.
  Za poljuben $\varepsilon > 0$ je $a_n < a + \varepsilon$ za neskončno mnogo $n
  \in \N$, saj bi bil sicer $a + \varepsilon \in U$.
  Ker pa je $a$ natančna zgornja meja, obstaja $u \in U$, ki je večji od $a -
  \varepsilon$, torej je $a_n < u$ le za končno indeksov $n$ in je na intervalu
  $(u, a + \varepsilon) \subseteq (a-\varepsilon, a+\varepsilon)$ neskončno
  mnogo členov zaporedja.
\end{proof}

\begin{definicija}
  Zaporedje je \pojem{naraščajoče}, če velja $a_n \le a_{n+1}$ za vse $n \in
  \N$.
  Je \pojem{strogo naraščajoče}, če velja $a_n < a_{n+1}$ za vse $n \in \N$.
  Analogno definiramo (strogo) padajoča zaporedja.
\end{definicija}

\begin{izrek}
  Naj bo $(a_n)_n$ monotono zaporedje (naraščajoče ali padajoče).
  Potem je $(a_n)_n$ konvergentno natanko tedaj, ko je omejeno.
\end{izrek}

\begin{proof}
  Implikacija v desno je natanko
  trditev~\ref{trditev:konvergentno-implicira-omejeno}.
  V levo:
  Brez škode za splošnost predpostavimo, da je $(a_n)_n$ naraščajoče in navzgor
  omejeno.
  Potem obstaja $a = \sup a_n$.
  Naj bo $\varepsilon > 0$.
  Potem po definiciji supremuma obstaja $n \in \N$, da je $a_n \in \oz{a -
	\varepsilon, a}$; ker pa je zaporedje naraščajoče, to velja tudi za vse
  kasnejše člene.
\end{proof}

\begin{definicija}
  Naj bo $(a_n)_n$ zaporedje realnih števil in $(n_k)_k$ strogo naraščajoče
  zaporedje naravnih števil.
  Zaporedju $(a_{n_k})_k$ pravimo \pojem{podzaporedje} zaporedja $(a_n)_n$.
\end{definicija}

\begin{trditev}
  Če $(a_n)_n$ konvergira k $L \in \R$, tja konvergira tudi poljubno
  podzaporedje $(a_n)_n$.
\end{trditev}

\begin{proof}
  Naj bo $\varepsilon > 0$ in $n_0$ tak, da za vse $n \ge n_0$ velja $\abs{a_n -
	L} < \varepsilon$.
  Naj bo $k_0 \in \N$ tak, da je $n_{k_0} \ge n_0$.
  Potem za vsak $k \ge k_0$ velja $\abs{a_{n_k} - L} < \varepsilon$.
\end{proof}

\begin{izrek}
  Število $s \in \R$ je stekališče zaporedja $(a_n)_n$ natanko tedaj, ko obstaja
  podzaporedje, ki konvergira k $s$.
\end{izrek}

\begin{proof}
  Implikacija v levo je očitna.
  V desno:
  Induktivno konstruirajmo podzaporedje $(a_{n_k})_k$, da bo za vse $k \in \N$
  veljalo $\abs{a_{n_k} - s} < 2^{-k}$.
  Za prvi člen vzemimo poljuben člen zaporedja, ki je od $s$ oddaljen za največ
  $\pol$.
  Recimo, da smo že izbrali $k-1$ členov.
  Ker je $s$ stekališče $(a_n)_n$, obstaja neskončno členov v $2^{-k}$-okolici
  $s$, torej obstaja tudi člen z indeksom, večjim od $n_{k-1}$.
  Ta člen izberemo za $a_{n_k}$.
  Po konstrukciji $(a_{n_k})_k$ konvergira k $s$.
\end{proof}

\begin{lema}
  \label{lema:primerjava-zaporedij}
  Če sta $(a_n)_n$ in $(b_n)_n$ konvergentni zaporedji, za kateri velja $a_n \le
  b_n$ za vse $n \in \N$, potem je $\lim a_n \le \lim b_n$.
\end{lema}

\begin{proof}
  Recimo nasprotno, da je $b := \lim b_n < a := \lim a_n$.
  Naj bo $\varepsilon = \inv{3} (a - b)$.
  Potem je zunaj $\varepsilon$-okolice $b$ le končno mnogo členov zaporedja
  $(b_n)_n$, in podobno zunaj $\varepsilon$-okolice $a$ le končno mnogo členov
  $(a_n)_n$.
  Potem za neskončno mnogo indeksov $n$ velja $b_n < a_n$, saj sta okolici
  disjunktni.
  To pa je protislovje s predpostavko.
\end{proof}

\begin{izrek}[Izrek o sendviču]
  Naj za zaporedja $(a_n)_n$, $(b_n)_n$ in $(c_n)_n$ velja $a_n \le b_n \le c_n$
  za vse $n \in \N$.
  Če sta $(a_n)_n$ in $(c_n)_n$ konvergentni ter je $\lim a_n = \lim c_n$, potem
  je tudi $(b_n)_n$ konvergentno z isto limito.
\end{izrek}

\begin{proof}
  Zaradi leme~\ref{lema:primerjava-zaporedij} je dovolj pokazati, da je
  $(b_n)_n$ konvergentno.
  Naj bo $\varepsilon > 0$.
  Označimo $L = \lim a_n = \lim c_n$.
  Obstaja $n_0 \in \N$, da za vse $n \ge n_0$ velja $a_n, c_n \in
  (L-\varepsilon,L+\varepsilon)$; ker pa je $a_n \le b_n \le c_n$, je tudi $b_n
  \in (L-\varepsilon, L+\varepsilon)$.
\end{proof}

% LocalWords:  Dedekindovimi Dedekindov Dedekindovemu Arhimedska Dedekindovega
% LocalWords:  Dedekindovem Dedekindove arhimedska supremuma
